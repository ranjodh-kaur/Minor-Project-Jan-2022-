\documentclass[english]{article}
\usepackage[T1]{fontenc}
\usepackage[latin9]{inputenc}
\usepackage[letterpaper]{geometry}
\geometry{verbose,tmargin=2.5cm,bmargin=1.25cm,lmargin=3.5cm,rmargin=1.25cm}
\usepackage{color}
\usepackage{float}
\usepackage{textcomp}
\usepackage{graphicx}
\usepackage{setspace}
\doublespacing

\makeatletter
\@ifundefined{date}{}{\date{}}
\makeatother

\usepackage{babel}
\begin{document}
\title{Website of Association of Computing Machinery for IT Department }
\author{MINOR PROJECT SYNOPSIS }
\maketitle
\begin{center}
\textbf{BACHELOR OF TECHNOLOGY }
\par\end{center}

\begin{doublespace}
\begin{center}
Information Technology
\par\end{center}
\end{doublespace}

\begin{center}
SUBMITTED BY
\par\end{center}

\begin{center}
AKSHAT SALUJA, ANMOL SINGH,ARSHDEEP SINGH
\par\end{center}

\begin{center}
University Roll no. 1905300,1905307,1905309
\par\end{center}

\begin{center}
\includegraphics[scale=0.2]{./logo_college.jpg}
\par\end{center}

\begin{center}
GURU NANAK DEV ENGINEERING COLLEGE
\par\end{center}

\begin{center}
LUDHIANA-141006, INDIA
\par\end{center}

\thispagestyle{empty} 

\newpage{}

\tableofcontents{}

\thispagestyle{empty} 

\newpage{}

\section{Introduction }

The Association for Computing Machinery (ACM) is a US-based international
learned society for computing. It was founded in 1947 and is the world's
largest scientific and educational computing society. The ACM is a
non-profit professional membership group, claiming nearly 100,000
student and professional members as of 2019. Its headquarters are
in New York City.

The field of project is Front-End-Web Development at first and later
Deployment of the web application. 

The project is to develop a website for ACM local chapter which comes
under IT department of GNDEC. Since the initial motive of the site
is to bring awareness as to what is ACM and display the names of the
students involved in local chapter therefore the main task is to build
a SPA (Single Page Application) which is minimal and responsive.

The initial idea is to use React with Tailwind for frontend since
React is widely known for developing user friendly and fast SPA\textquoteright s.
Tailwind is a utility-first CSS framework which makes beautiful UIs
with the help of powerful classes that can be used with the tags itself.

\quad{} 

\setcounter{page}{1}

\newpage{}

\section{Rationale}

Few reasons for making a website for ACM:

\textbullet{} General Awareness: What is ACM local chapter, its current
members and their role

\textbullet{} Information Exchange: upcoming and past events

\textbullet{} Online Presence: making all the info related to society
available online

\textbullet{} Gather more members\newpage{}

\section{Feasibility Study }

\textbf{Financial Feasibility:}

Being a website, it will have an associated hosting cost. Since the
perspective users are students of the college only and the website
doesn\textquoteright t require any multimedia file transfer between
client and server therefore very less bandwidth and storge is required.
Due to the above reasons the website can be easily hosted at the server
at the campus which is provided by the college hence no cost is associated
with hosting of the website.

From this it\textquoteright s clear that the project is financially
feasible.

\noindent \textbf{Technical Feasibility:}

The ACM website is a SPA. The main tools and technologies required
for developing such a website are:

\textbullet{} HTML 

\textbullet{} CSS 

\textbullet{} JS 

\textbullet{} React 

\textbullet{} Tailwind 

\textbullet{} Node 

\textbullet{} Git 

\textbullet{} Ide like vs code 

\textbullet{} Web Browser 

All of the technologies and tools listed above are open source and
free to use. The skill required to use these is manageable. Therefore,
the project is also technically feasible. 

\newpage{}

\section{Methodology/ \textcolor{black}{Planning} of work }

For doing basic research on the project, we can look for other ACM
(local chapter) websites depending upon this weekend include various
sections into our own website. Since the final site involves many
pages and components these can be divided among the team members and
later integrated. The components will be designed in a manner such
that they can be reused to avoid code duplication also keeping the
overall size of the site small. The data i.e., name of the members
of the society that are to be displayed on the main website can be
collected via a Google form. Since multiple people will be working
upon the project therefore it is necessary to you use a version control
system. For tracking the changes overtime, git will be used. GitHub
will be used as a remote server to which each developer working on
the project can push and pull changes. At last deadlines would be
set for achieving goals and timely completion of the project. 

\newpage{}

\section{Facilities required for proposed work }

\textbf{Hardware:} Basic requirements : A pc or laptop having efficient
RAM and HDD (Intel Core i3 or equivalent , 20 GB free space on HDD
for install various software), any operating system (windows, Linux,
MacOS) installed with latest any web browsers which supports modern
JavaScript (Chrome, Mozilla Firefox) and also need a text editor such
as notepad, notepad++, sublime text etc. From the hardware point of
view just needs a basic setup.\\

\noindent \textbf{Software:} Installation and setup of following: 

\textbullet{} Visual Studio Code or any other IDE 

\textbullet{} Node Js

\textbullet{} NPM 

\textbullet{} Adding path 

\newpage{}

\section{Objectives }

The main objectives of this project are: 
\begin{enumerate}
\item To make a functional, informative and appealing website
\item To take registrations for upcoming events through website
\end{enumerate}
\newpage{}

\section{Outcomes}

\noindent \textbullet{} Develop Front-End-Develop skills 

\noindent \textbullet{} Able to use frameworks like Tailwind CSS to
design UI 

\noindent \textbullet{} Use React to make website faster reducing
the time to fetch html pages again 

\noindent \textbullet{} Able to use Git to track changes and collaborate
on a project 

\noindent \textbullet{} Deploy the website in a production environment
\newpage{}

\section{References}
\begin{itemize}
\item ``https://en.wikipedia.org/wiki/Association\_for\_Computing\_Machinery''.This
page was last edited on 22 March 2022, at 04:12 (UTC).
\item ``http://robin.bilkent.edu.tr/lyxguide.pdf'',Robin Turner 22nd February
2001. Last accessed on 6th April 2022.
\item ``https://www.acm.org/about-acm/about-the-acm-organization'', Last
accessed on 6th April 2022.
\item ``https://www.slideshare.net/PasinduTennage/sample-software-engineering-feasibility-study-report'',
Last accessed on 6th April 2022.
\item ``https://tailwindcss.com/docs/utility-first'', Last accessed on
6th April 2022.
\end{itemize}
\newpage
\end{document}
